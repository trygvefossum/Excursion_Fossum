Let us in particular consider a locally finite (Borel) measure $\mes$ on $D$ and investigate the probability distribution 
of $\mes(\es)$ through its moments.

\medskip

%That $\mu(\es)$ is a well-defined random variable follows from
\begin{propo}
    \label{propo1}
%Under technical conditions to be precised, $\mu(\es)$ possesses moments or arbitrary order and these can be expressed for any $r\geq 1$ as
For a measurable random field $\gp$ and a locally finite measure $\mes$ on $D$, $\mes(\es)$ is a random variable and for 
any $r\geq 1$,
\begin{equation*}
\begin{split}
\mathbb{E}[\mes(\es)^r]
&=\int_{D^{r}} \jointExcuProb
\productMeasure
,
\end{split}
\end{equation*}

where the product measure is denoted as
$\nu^{\otimes}:=\bigotimes_{i=1}^r \nu$.
%Note that we will use boldface to denote concatenated variables. % TO RELOCATE
Here $\gp$ is defined on $D$, and for
$\bm{u}=\left(u^{(1)}, ..., u^{(r)}\right)\in D^r$, $\gp[\bm{u}]=\left(\gp[u^{(1)}], ...,
\gp[u^{(r)}]\right)\in \mathbb{R}^{\no r}$.
\medskip

In the particular case where $\gp$ is a multivariate Gaussian random field
%and denoting the mean and covariance matrix of $\gp[\bm{u}]$ by $\meanUU\in\mathbb{R}^{p\times r}$ and $\covUU$, respectively,
we have
\begin{align*}
\jointExcuProb = \mathcal{N}_{\no r}(T^r; \meanUU, \covUU),
\end{align*}
where $\mathcal{N}_{\no r}(\cdot ; \meanUU, \covUU)$ is the Gaussian measure on $\mathbb{R}^{\no r}$ with mean $\meanUU$ 
and covariance matrix $\covUU$, respectively defined blockwise by
%where both are defined blockwise by
\begin{align*}
\meanUU&=\begin{pmatrix}\mu(u^{(1)})\\ \vdots\\ \mu(u^{(r)})\end{pmatrix}
\in \mathbb{R}^{\no r}, \\
\text{and } \covUU &= \begin{pmatrix}
\cov(\gp[u^{(1)}], \gp[u^{(1)}]) & \dots & \cov(\gp[u^{(1)}],
\gp[u^{(r)}])\\
\vdots & & \vdots\\
\cov(\gp[u^{(r)}], \gp[u^{(1)}]) & \dots & \cov(\gp[u^{(r)}],
\gp[u^{(r)}])\\
\end{pmatrix}\in \mathbb{R}^{pr\times pr},
\end{align*}
each of the $r\times r$ blocks of the latter matrix being itself a (cross-)covariance matrix of dimension $\no \times 
\no$.
%dimensional Gaussian random vector.
Assuming further that $\covUU$ is non-singular, the probability of interest can be formulated in terms of the $\no 
r$-dimensional Gaussian probability density function
$\varphi_{\no  r}(\cdot;~\meanUU, \covUU)$ as
%$\varphi_{\no \times r}$ as
%$\bm{u}\in D^r$ such that $\gp[\bm{u}]$ be non-degenerate, the integrand can be expressed in terms of the $r \times \no$-dimensional Gaussian distribution, namely
%cumulative distribution function, namely
\begin{equation*}
\begin{split}
&\jointExcuProb
=
\int_{T^r} \varphi_{\no  r}\left(\bm{v};
    ~\meanUU, \covUU\right)
    \mathrm{d}\bm{v},
\end{split}
\end{equation*}
In the particular orthant case with $\T=(-\infty, t_1] \times \dots \times (-\infty, t_{r}]$,
the latter probability directly writes in terms of the multivariate Gaussian
cumulative distribution, % with mean $\meanUU$ and covariance matrix $\covUU$,
this time by the way without requiring $\covUU$ %the latter
to be non-singular:
\begin{equation*}
\begin{split}
\jointExcuProb
&=
\varPhi_{\no r}\left(\bm{t};~\meanUU, \covUU\right),
\end{split}
\end{equation*}
where we have used the notations
$t=(t_1,\dots,t_{\no}%,..., t_1, ..., t_p
)\in\mathbb{R}^{\no}$, $1_{r}=(1,\dots,1)\in \R^{r}$, and
$\bm{t}=1_{r}\otimes \bm{t}=(t_1,\dots,t_{\no},\dots,t_1,\dots,t_{\no})
\in \R^{\no r}$.
\end{propo}
\begin{proof}
That $\mes(\es)$ defines indeed a random variable follows from Fubini's theorem
relying on the joint measurability of
$(\x, \omega) \to \mathbbm{1}_{\es(\omega)}(\x)$,
%= \mathbf{1}_{\xi_{\x}(\omega) \in T}$ and Fubini's theorem.
itself inherited from the assumed measurability for
$(\x, \omega) \to \gp[\x](\omega)$ and $T$, respectively. From there, following the steps of Robbins' theorem \cite{Robins1944}, we find that
\begin{equation*}
\begin{split}
\mathbb{E}[\mes(\es)^{r}]
&=\mathbb{E}\left[\left(\int_{D} \mathbbm{1}_{\gp[u] \in T} ~d\mes(u) \right)^{r} \right]
=\mathbb{E}\left[ \prod_{i=1}^{r} \left(
        \int_{D} \mathbbm{1}_{\gp[\uu^{(i)}] \in T} ~d\mes(\uu^{(i)})
\right) \right] \\
&
=
%\mathbb{E}\left[
%\int_{D^{r}}
%\prod_{i=1}^{r}
%\mathbf{1}_{\gp[\uu^{(i)}] \in T}
%\productMeasure
%\right]
%=
\mathbb{E}\left[
\int_{D^{r}}
\mathbbm{1}_{\gp[\uu^{(1)}] \in T,\dots, \gp[\uu^{(r)}]  \in T}
\productMeasure
\right]
%\\
%&=\int_{D^{r}} \mathbb{E}\left[\mathbf{1}_{\gp[\bm{u}] \in T^r} \right] \productMeasure \\
%&
=\int_{D^{r}}
\jointExcuProb
\productMeasure.
\end{split}
\end{equation*}
The rest consists in expliciting the probability of $T\times \dots \times T$ under the multivariate Gaussian distribution of
$\left(\gp[u^{(1)}], \dots,  \gp[u^{(r)}] \right)$.
\end{proof}

Proposition~\ref{propo1} enables in turn calculating centred moments of $\mes(\es)$.
The excursion measure variance $\emv = \operatorname{Var}[\mes(\es)]$ is probably the most relevant of them for the considered purposes, and hence writes
\begin{equation*}
\begin{split}
\emv
&=\int_{D^2} \mathbb{P}\left(
\gp[u]\in T, \gp[v]\in T \right)
d\mes^{\otimes}(u, v)\\
&-\left( \int_{D} \mathbb{P}\left(\gp[u]\in T\right) d\mes(u) \right)^2,
\end{split}
\end{equation*}
which boils down in the excursion/sojourn case where $\T=(-\infty, t_1] \times
\dots \times (-\infty, t_{\no}]$ to
\begin{equation*}
\begin{split}
\emv
%\operatorname{Var}[\mes(\es)]
&=\int_{D^2}
\varPhi_{2\no}
\left(
(\bt, \bt); \mu((u,v)),
K((u,v),(u,v))
\right)
\
\mathrm{d}\mes^{\otimes} %\mes
%\productMeasure
(u,v)\\
&-\left( \int_{D} \varPhi_{\no}\left(\bt;\mu(u), K(u)\right) d\mes(u) \right)^2,
\end{split}
\end{equation*}
%
Note that like in the case of scalar-valued fields, this quantity requires to work out an integral over $D^2$. In 
contrast, and still like in the scalar-valued case, the Integrated Bernoulli Variance (IBV) of \cite{Bect.etal} involves 
solely and integral on $D$ and can be expanded in our present settings as follows
\begin{equation*}
\begin{split}
\operatorname{IBV} %(\es) %;\mes)
&=\int_{D}
\mathbb{P}\left(\gp[\uu]\in T\right)(1-\mathbb{P}\left(\gp[\uu]\in T\right))
d\mes(u) \\
&=\int_{D}
\varPhi_{\no}\left(\bt;\mu(\uu), K(\uu)\right)
-\left(\varPhi_{\no}\left(\bt;\mu(\uu), K(\uu)\right) \right)^2
\mathrm{d}\mes(u).
\end{split}
\end{equation*}
%
Next we investigate uncertainty reduction criteria related to those indicators.
In order to be able to calculate them in semi-analytical forms relying on integrals of multivariate Gaussian cumulative 
distribution functions, we will appeal to some key intermediate results presented next.