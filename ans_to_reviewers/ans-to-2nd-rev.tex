\documentclass[a4paper]{article}
\topmargin	-10mm
\oddsidemargin	0mm
\textwidth	150mm
\textheight	230mm

\newcommand{\ba}{ {\boldsymbol a} }
\newcommand{\bA}{ {\boldsymbol A} }
\newcommand{\bb}{ {\boldsymbol b} }
\newcommand{\bB}{ {\boldsymbol B} }
\newcommand{\bc}{ {\boldsymbol c} }
\newcommand{\bC}{ {\boldsymbol C} }
\newcommand{\bd}{ {\boldsymbol d} }
\newcommand{\bD}{ {\boldsymbol D} }
\newcommand{\be}{ {\boldsymbol e} }
\newcommand{\bE}{ {\boldsymbol E} }
\newcommand{\boldf}{ {\boldsymbol f} }
\newcommand{\bF}{ {\boldsymbol F} }
\newcommand{\bg}{ {\boldsymbol g} }
\newcommand{\bG}{ {\boldsymbol G} }
\newcommand{\bh}{ {\boldsymbol h} }
\newcommand{\bH}{ {\boldsymbol H} }
\newcommand{\bi}{ {\boldsymbol i} }
\newcommand{\bI}{ {\boldsymbol I} }
\newcommand{\bj}{ {\boldsymbol j} }
\newcommand{\bJ}{ {\boldsymbol J} }
\newcommand{\bk}{ {\boldsymbol k} }
\newcommand{\bK}{ {\boldsymbol K} }
\newcommand{\bl}{ {\boldsymbol l} }
\newcommand{\bL}{ {\boldsymbol L} }
\newcommand{\bm}{ {\boldsymbol m} }
\newcommand{\bM}{ {\boldsymbol M} }
\newcommand{\bn}{ {\boldsymbol n} }
\newcommand{\bN}{ {\boldsymbol N} }
\newcommand{\bo}{ {\boldsymbol o} }
\newcommand{\bO}{ {\boldsymbol O} }
\newcommand{\bp}{ {\boldsymbol p} }
\newcommand{\bP}{ {\boldsymbol P} }
\newcommand{\bq}{ {\boldsymbol q} }
\newcommand{\bQ}{ {\boldsymbol Q} }
\newcommand{\br}{ {\boldsymbol r} }
\newcommand{\bR}{ {\boldsymbol R} }
\newcommand{\bs}{ {\boldsymbol s} }
\newcommand{\bS}{ {\boldsymbol S} }
\newcommand{\bt}{ {\boldsymbol t} }
\newcommand{\bT}{ {\boldsymbol T} }
\newcommand{\bu}{ {\boldsymbol u} }
\newcommand{\bU}{ {\boldsymbol U} }
\newcommand{\bv}{ {\boldsymbol v} }
\newcommand{\bV}{ {\boldsymbol V} }
\newcommand{\bw}{ {\boldsymbol w} }
\newcommand{\bW}{ {\boldsymbol W} }
\newcommand{\bx}{ {\boldsymbol x} }
\newcommand{\bX}{ {\boldsymbol X} }
\newcommand{\by}{ {\boldsymbol y} }
\newcommand{\bY}{ {\boldsymbol Y} }
\newcommand{\bz}{ {\boldsymbol z} }
\newcommand{\bZ}{ {\boldsymbol Z} }
\newcommand{\vc}[1]{\mbox{\boldmath $1$}}
\newcommand{\balph}{ {\boldsymbol \alpha} }
\newcommand{\balpha}{ {\boldsymbol \alpha} }
\newcommand{\bbet}{ {\boldsymbol \beta} }
\newcommand{\bbeta}{ {\boldsymbol \beta} }
\newcommand{\bgam}{ {\boldsymbol \gamma} }
\newcommand{\bgamma}{ {\boldsymbol \gamma} }
\newcommand{\bGamma}{ {\boldsymbol \Gamma} }
\newcommand{\bdelta}{ {\boldsymbol \delta} }
\newcommand{\bDelta}{ {\boldsymbol \Delta} }
\newcommand{\beps}{ {\boldsymbol \epsilon} }
\newcommand{\bepsilon}{ {\boldsymbol \epsilon} }
\newcommand{\bphi}{ {\boldsymbol \phi} }
\newcommand{\bPhi}{ {\boldsymbol \Phi} }
\newcommand{\bpi}{ {\boldsymbol \pi} }
\newcommand{\bpsi}{ {\boldsymbol \psi} }
\newcommand{\bkap}{ {\boldsymbol \kappa} }
\newcommand{\bkappa}{ {\boldsymbol \kappa} }
\newcommand{\bKappa}{ {\boldsymbol \Kappa} }
\newcommand{\blam}{ {\boldsymbol \lambda} }
\newcommand{\blambda}{ {\boldsymbol \lambda} }
\newcommand{\bLambda}{ {\boldsymbol \Lambda} }
\newcommand{\bmu}{ {\boldsymbol \mu} }
\newcommand{\bMu}{ {\boldsymbol \Mu} }
\newcommand{\bet}{ {\boldsymbol \eta} }
\newcommand{\bome}{ {\boldsymbol \omega} }
\newcommand{\bomega}{ {\boldsymbol \omega} }
\newcommand{\bOmega}{ {\boldsymbol \Omega} }
\newcommand{\bnabla}{ {\boldsymbol \nabla} }
\newcommand{\brho}{ {\boldsymbol \rho} }
\newcommand{\bsigma}{ {\boldsymbol \sigma} }
\newcommand{\bSig}{ {\boldsymbol \Sigma} }
\newcommand{\bSigma}{ {\boldsymbol \Sigma} }
\newcommand{\btheta}{ {\boldsymbol \theta} }
\newcommand{\bTheta}{ {\boldsymbol \Theta} }
\newcommand{\bzeta}{ {\boldsymbol \zeta} }
\newcommand{\bPsi}{ {\boldsymbol \Psi} }
\newcommand{\btau}{ {\boldsymbol \tau} }
\newcommand{\bzero}{ {\boldsymbol 0} }
\newcommand{\bones}{ {\boldsymbol 1} }
\newcommand{\given}{\,|\,}
\newcommand{\sS}{{\cal S}}
\newcommand{\Ss}{{\cal S}}
\newcommand{\Field}{{\cal F}}
\newcommand{\colsp}{{\cal C}}
\newcommand{\nullsp}{{\cal N}}
\newcommand{\rowsp}{{\cal R}}
\newcommand{\tildeC}{\tilde{C}}
\newcommand{\tildeK}{\tilde{K}}
\newcommand{\tildew}{\tilde{w}}
\newcommand{\tildebw}{\tilde{\bw}}
\newcommand{\tildebW}{\tilde{\bW}}
\newcommand{\calC}{{\cal C}}
\newcommand{\calcbC}{{\bf {\cal C}}}

\def\rx{{\texttt{T-REX\ }}}
\def\rxe{{\texttt{T-REX}}}
\def\du{{\texttt{DUNE}\ }}
\def\due{{\texttt{DUNE}}}

% == Maths
\usepackage{amssymb}
\usepackage{amsmath}
\usepackage{graphicx}
\usepackage{enumitem}
\usepackage{xcolor}
\usepackage[normalem]{ulem} %to strike the words

\newcounter{reviewer}
\def\revcom{\textbf{Reviewer comment}}
\def\aecom{\textbf{Associate Editor comment }}
\def\reply{\textbf{Reply}}
\def\action{\textbf{Action}}

\newlist{answers}{enumerate}{10}
\setlist[answers]{leftmargin=*, label=\arabic{reviewer}.\arabic*}

\newcommand{\edcomment}[1]{{\color{green}{\{Editor: 1\}}}}
\newcommand{\frevcomment}[1]{{\color{blue}{\{Rev 1: 1\}}}}
\newcommand{\srevcomment}[1]{{\color{red}{\{Rev 2: 1\}}}}
\newcommand{\trevcomment}[1]{{\color{violet}{\{Rev 3: 1\}}}}

\usepackage{amsthm,amsmath,natbib}
\RequirePackage[colorlinks,citecolor=blue,urlcolor=blue]{hyperref}

\begin{document}

%----------------------------------------------------------------------
\title{Autonomous Oceanographic Data Collection. Learning Excursion Sets of Vector-valued Gaussian Random Fields
  \\\vspace{5mm}
 Authors response to 2$^{nd}$ review
  \\\vspace{5mm}
\small{Manuscript: AOAS1910-027}}
\author{ }

\date{\today}

\maketitle

We are grateful to the editors and reviewers for their comments and
suggestions. They have all been considered in this revised
version of our manuscript.\\

\par \vspace{1em}


%======================================================================
\section*{Editor}
%======================================================================
All comments of the Editor are listed below and answered on a point-by-point basis.

\vspace{5mm}
\noindent {\bf{1. Sampling limitations:}} You explained in your response file (R1S3.pdf) the limitations of the sampling design (comment #4 from the AE's previous review). I agree with the AE (bullet 1, R2AE2.pdf) that some (perhaps slightly abbreviated) version of your response on p4 (middle) would be important to include; viz.

\begin{quote}
"The sampling capabilities are restricted by the sensors onboard
the AUV, the computational load and navigation resources. The sampling
of temperature and salinity variables can be at 1Hz (higher than the
grid resolution), but, due to computational and navigation resources
available on the AUV, this simplification of data assimilation at the
waypoints is necessary."   
\end{quote}


\reply: We agree and have addressed this point in \aecom \textbf{1 - Reply to AE comment 4}. 
 
\vspace{5mm}
\noindent {\bf{2. Variance of errors:}} In view of the apparent nugget effects of 0.42 and 0.7 for salinity
and temperature, respectively, in Fig 9(c) (p21), I confess to being
a little puzzled myself by this statement on p22:
"We specify variance $0.25^2$ for both errors, which is based on a
middle ground between the nugget effect in the empirical variogram
and the sensor specifications."  Is 0.0625 really a "middle ground"?
Or, did you have other reasons for choosing 0.0625 not stated here?
In any event, I agree with the AE (bullet 5): "either give a better
justification of your choice of 0.0625 or, if that’s not possible,
then redo the analysis with a more justifiable choice."

\vspace{5mm}
\reply:

\vspace{5mm}
\noindent {\bf{3. Exposition:}} Apart from corrections noted by the AE (bullets 2,3,4,6), Referee 1
offered suggestions on shortening. Neither the AE nor I feel that re-structuring is needed.  In fact, some of the Referee's proposed changes would not be even desirable, from my perspective: I would not want you to "Consider putting the background material in an appendix to go as supplementary material" because, in my opinion, this background material is essential. And the simulation section (design and results) is already quite efficient. I include the Referee's report (R2R5.pdf) primarily for your information (the remarks from Ehrenberg are worth reading).

\vspace{5mm}
\reply: 

\vspace{5mm}
\noindent {\bf{4. Figures:}} Like most journals, AOAS discourages the use of color in figures,
unless absolutely essential for reader comprehension.  I believe
that color is important for Figures 1, 2, 4, 11 and perhaps 5 and 6.
But color is not even mentioned in the caption for Figure 3; your
different line types likely will suffice in Figures 8 and 9; and
probably nothing will be lost if Figures 7 & 10 were black-and-white.
Color is very expensive to print, so we appreciate your attention
to this cost-saving measure.

\vspace{5mm}
\reply: Line type will be tricky for Fig. 8, when lines are very close together especially in Fig.8b, and Fig.8c. 

\vspace{5mm}
\noindent {\bf{Other comments:}} In your revision, you may want to include in the
acknowledgements, p26, the conscientious efforts of the anonymous
reviewers, whose comments led to this favorable outcome.

\vspace{5mm}
\reply: 

%======================================================================
 \section*{Associate Editor}
%======================================================================
All comments of the Associate Editor are listed below and answered on a point-by-point basis.

\setcounter{reviewer}{1}

\vspace{5mm}
\noindent \aecom \textbf{1 - Reply to AE comment 4}: 
I hank you for explaining the sampling limitations to me in your reply.  However, I think these limitations should also be noted in the narrative, and the sub-optimality of the resulting trajectory should be acknowledged.

\vspace{5mm}
\reply: We agree that this should be mentioned in the text. We have added the following text in the Closing remarks section at page 25.

\begin{quote}
    "The sampling capabilities are restricted by the sensors onboard the AUV, the computational load and navigation resources. The sampling of temperature and salinity variables can be much higher than the grid resolution, but, due to computational and navigation resources available on the AUV, assimilation into a discretized fixed grid is necessary.``
\end{quote}

\vspace{5mm}
\noindent \aecom \textbf{2 - Page 2, line 15}:  I think it would be more appropriate to refer to this as sampling design rather than sampling.

\vspace{5mm}
\reply: We have removed the word sampling from the sentence, with made it more clear and readable. See the change below:

\begin{quote}
    "Models and methods from spatial statistics and experimental design can clearly contribute to this challenge.``
\end{quote}

\vspace{5mm}
\noindent \aecom \textbf{3 - Page 4, line 3}:  Replace “proxys” with the correct plural “proxies”

\vspace{5mm}
\reply: Changed to “proxies” on page 4.

\vspace{5mm}
\noindent \aecom \textbf{4 - Page 5, line 16:}  Change  “the  PhD  thesis  (Stroh,  2018,  p.  82)”  to ”Stroh (2018, p. 82)”

\vspace{5mm}
\reply: We have changed the reference accordingly in the text at Page 5. 

\vspace{5mm}
\noindent \aecom \textbf{5 - Page 22, line 8:}  From Figure 9(c), the nugget effect in the empirical variogram appears to be approximately 0.4 for salinity and 0.7 for temperature.  Even if the sensor specifications were essentially 0, it would be a stretch to say that a variance of 0.0625 was a “middle ground” between the empirical nugget effect and the sensor specifications.  So either give a better justification of your choice of 0.0625 or, if that’snot possible, then redo the analysis with a more justifiable choice.

\vspace{5mm}
\reply:

\vspace{5mm}
\noindent \aecom \textbf{6 - Page 22, line 5:}  Insert “triangular” immediately after “equilateral”

\vspace{5mm}
\reply:

\vspace{1em}


% == Adding references
\footnotesize
\bibliographystyle{apalike}
\bibliography{ref}

%=====================================================================
\end{document}
%=====================================================================
