\documentclass[a4paper]{article}
\topmargin	-10mm
\oddsidemargin	0mm
\textwidth	150mm
\textheight	230mm

\newcommand{\ba}{ {\boldsymbol a} }
\newcommand{\bA}{ {\boldsymbol A} }
\newcommand{\bb}{ {\boldsymbol b} }
\newcommand{\bB}{ {\boldsymbol B} }
\newcommand{\bc}{ {\boldsymbol c} }
\newcommand{\bC}{ {\boldsymbol C} }
\newcommand{\bd}{ {\boldsymbol d} }
\newcommand{\bD}{ {\boldsymbol D} }
\newcommand{\be}{ {\boldsymbol e} }
\newcommand{\bE}{ {\boldsymbol E} }
\newcommand{\boldf}{ {\boldsymbol f} }
\newcommand{\bF}{ {\boldsymbol F} }
\newcommand{\bg}{ {\boldsymbol g} }
\newcommand{\bG}{ {\boldsymbol G} }
\newcommand{\bh}{ {\boldsymbol h} }
\newcommand{\bH}{ {\boldsymbol H} }
\newcommand{\bi}{ {\boldsymbol i} }
\newcommand{\bI}{ {\boldsymbol I} }
\newcommand{\bj}{ {\boldsymbol j} }
\newcommand{\bJ}{ {\boldsymbol J} }
\newcommand{\bk}{ {\boldsymbol k} }
\newcommand{\bK}{ {\boldsymbol K} }
\newcommand{\bl}{ {\boldsymbol l} }
\newcommand{\bL}{ {\boldsymbol L} }
\newcommand{\bm}{ {\boldsymbol m} }
\newcommand{\bM}{ {\boldsymbol M} }
\newcommand{\bn}{ {\boldsymbol n} }
\newcommand{\bN}{ {\boldsymbol N} }
\newcommand{\bo}{ {\boldsymbol o} }
\newcommand{\bO}{ {\boldsymbol O} }
\newcommand{\bp}{ {\boldsymbol p} }
\newcommand{\bP}{ {\boldsymbol P} }
\newcommand{\bq}{ {\boldsymbol q} }
\newcommand{\bQ}{ {\boldsymbol Q} }
\newcommand{\br}{ {\boldsymbol r} }
\newcommand{\bR}{ {\boldsymbol R} }
\newcommand{\bs}{ {\boldsymbol s} }
\newcommand{\bS}{ {\boldsymbol S} }
\newcommand{\bt}{ {\boldsymbol t} }
\newcommand{\bT}{ {\boldsymbol T} }
\newcommand{\bu}{ {\boldsymbol u} }
\newcommand{\bU}{ {\boldsymbol U} }
\newcommand{\bv}{ {\boldsymbol v} }
\newcommand{\bV}{ {\boldsymbol V} }
\newcommand{\bw}{ {\boldsymbol w} }
\newcommand{\bW}{ {\boldsymbol W} }
\newcommand{\bx}{ {\boldsymbol x} }
\newcommand{\bX}{ {\boldsymbol X} }
\newcommand{\by}{ {\boldsymbol y} }
\newcommand{\bY}{ {\boldsymbol Y} }
\newcommand{\bz}{ {\boldsymbol z} }
\newcommand{\bZ}{ {\boldsymbol Z} }
\newcommand{\vc}[1]{\mbox{\boldmath $#1$}}
\newcommand{\balph}{ {\boldsymbol \alpha} }
\newcommand{\balpha}{ {\boldsymbol \alpha} }
\newcommand{\bbet}{ {\boldsymbol \beta} }
\newcommand{\bbeta}{ {\boldsymbol \beta} }
\newcommand{\bgam}{ {\boldsymbol \gamma} }
\newcommand{\bgamma}{ {\boldsymbol \gamma} }
\newcommand{\bGamma}{ {\boldsymbol \Gamma} }
\newcommand{\bdelta}{ {\boldsymbol \delta} }
\newcommand{\bDelta}{ {\boldsymbol \Delta} }
\newcommand{\beps}{ {\boldsymbol \epsilon} }
\newcommand{\bepsilon}{ {\boldsymbol \epsilon} }
\newcommand{\bphi}{ {\boldsymbol \phi} }
\newcommand{\bPhi}{ {\boldsymbol \Phi} }
\newcommand{\bpi}{ {\boldsymbol \pi} }
\newcommand{\bpsi}{ {\boldsymbol \psi} }
\newcommand{\bkap}{ {\boldsymbol \kappa} }
\newcommand{\bkappa}{ {\boldsymbol \kappa} }
\newcommand{\bKappa}{ {\boldsymbol \Kappa} }
\newcommand{\blam}{ {\boldsymbol \lambda} }
\newcommand{\blambda}{ {\boldsymbol \lambda} }
\newcommand{\bLambda}{ {\boldsymbol \Lambda} }
\newcommand{\bmu}{ {\boldsymbol \mu} }
\newcommand{\bMu}{ {\boldsymbol \Mu} }
\newcommand{\bet}{ {\boldsymbol \eta} }
\newcommand{\bome}{ {\boldsymbol \omega} }
\newcommand{\bomega}{ {\boldsymbol \omega} }
\newcommand{\bOmega}{ {\boldsymbol \Omega} }
\newcommand{\bnabla}{ {\boldsymbol \nabla} }
\newcommand{\brho}{ {\boldsymbol \rho} }
\newcommand{\bsigma}{ {\boldsymbol \sigma} }
\newcommand{\bSig}{ {\boldsymbol \Sigma} }
\newcommand{\bSigma}{ {\boldsymbol \Sigma} }
\newcommand{\btheta}{ {\boldsymbol \theta} }
\newcommand{\bTheta}{ {\boldsymbol \Theta} }
\newcommand{\bzeta}{ {\boldsymbol \zeta} }
\newcommand{\bPsi}{ {\boldsymbol \Psi} }
\newcommand{\btau}{ {\boldsymbol \tau} }
\newcommand{\bzero}{ {\boldsymbol 0} }
\newcommand{\bones}{ {\boldsymbol 1} }
\newcommand{\given}{\,|\,}
\newcommand{\sS}{{\cal S}}
\newcommand{\Ss}{{\cal S}}
\newcommand{\Field}{{\cal F}}
\newcommand{\colsp}{{\cal C}}
\newcommand{\nullsp}{{\cal N}}
\newcommand{\rowsp}{{\cal R}}
\newcommand{\tildeC}{\tilde{C}}
\newcommand{\tildeK}{\tilde{K}}
\newcommand{\tildew}{\tilde{w}}
\newcommand{\tildebw}{\tilde{\bw}}
\newcommand{\tildebW}{\tilde{\bW}}
\newcommand{\calC}{{\cal C}}
\newcommand{\calcbC}{{\bf {\cal C}}}

\def\rx{{\texttt{T-REX\ }}}
\def\rxe{{\texttt{T-REX}}}
\def\du{{\texttt{DUNE}\ }}
\def\due{{\texttt{DUNE}}}

% == Maths
\usepackage{amssymb}
\usepackage{amsmath}
\usepackage{graphicx}
\usepackage{enumitem}
\usepackage{xcolor}
\usepackage[normalem]{ulem} %to strike the words

\newcounter{reviewer}
\def\revcom{\textbf{Reviewer comment}}
\def\aecom{\textbf{Associate Editor comment}}
\def\reply{\textbf{Reply}}
\def\action{\textbf{Action}}

\newlist{answers}{enumerate}{10}
\setlist[answers]{leftmargin=*, label=\arabic{reviewer}.\arabic*}

\newcommand{\edcomment}[1]{{\color{green}{\{Editor: #1\}}}}
\newcommand{\frevcomment}[1]{{\color{blue}{\{Rev 1: #1\}}}}
\newcommand{\srevcomment}[1]{{\color{red}{\{Rev 2: #1\}}}}
\newcommand{\trevcomment}[1]{{\color{violet}{\{Rev 3: #1\}}}}

\usepackage{amsthm,amsmath,natbib}
\RequirePackage[colorlinks,citecolor=blue,urlcolor=blue]{hyperref}

\begin{document}

%----------------------------------------------------------------------
\title{Informative Autonomous Sampling of Oceanographic Variables Using Joint Excursion Sets
  \\\vspace{5mm}
 Authors response
  \\\vspace{5mm}
\small{Manuscript: AOAS1910-027}}
\author{ }

\date{\today}

\maketitle

We are grateful to the reviewers for their comments and
suggestions. They have all been considered in this revised
version of this manuscript. The principal revisions made to the manuscript are as follows and are done to address the main comments by Editors and Reviewers:

\begin{enumerate}
\item The motivation for excursion sets (ESs) and in particular that of {\bf{multivariate ESs}} has been strengthened. Motivating examples are abundant in medicine where doctors do not solely rely on a single symptom but must see several combined effects before making a diagnosis. Another example is that of photosynthesis where plants and algae produce oxygen only when given enough of every component; light, carbon dioxide and water. Our motivation in this paper is that temperature and salinity, combined, drive several processes in the oceans. For instance, salmon lice larvae are known to depend on relatively high temperature and high salinities to survive. Salmon lice are parasites to salmon, and the bivariate ES for temperature and salinity is hence important to understand for sustainable activities in salmon fisheries.

{\it{Action:}} In the revised manuscript these aspects are largely rewritten in the Introduction, Background, ES definitions and Case Study. This change addresses the comments of AE1, R1 (3,) MORE.

\item The {\bf{statistical methodological contribution}} have been clarified and improved, including definitions of ESs and having more general results related to the integrated Bernoulli variance (IBV) for multivariate processes, using marginal (one-location) entries and for volume calculations requiring multiple locations. The revised version highlights these novel statistical results as a main contribution of the paper. 

{\it{Action:}} In the revised manuscript these aspects are largely rewritten in the main body containing novel results for IBV for sums of marginal variances as well as for volumes, while the new Appendix derives the closed form results. This change addresses the comments of E2, AE1, R2 (7), R3 (1), MORE.

\item We realized that a contribution of the current work is available code for statisticians who could try this themselves or even extend it for their own statistical cases. The computer code could of course also be useful to practitioners having related sampling design problems. The {\bf{simulation results and code}} are thoroughly re-worked. Open-source python code is now provided for all the examples used in the paper and also for simulation studies using more nuances statistical models (non-stationary, non-isotropic and non-separable). 

{\it{Action:}} The simulation study section is largely modified and it contains pointers to the associated open-source code. This change addresses the comments of R1 (2,3),  MORE.


\item In the initial submission we tried to motivate these kinds of applications carefully, but based on the comments, we see that the {\bf{introduction and background}} were too long. We were not selling the paper enough to professional statisticians, and maybe too much to other scientists who are interested in particular areas of these statistical application. In the revised version we motivate more for statisticians, in particular we emphasize the statistical design problem which is sought solved by AUV with its constraints on navigation and computing resources.

{\it{Action:}} The Introduction and Background sections have been largely rewritten. We now get quicker to the point and engage readers in its both applied statistical focus and the methodological contributions. This change addresses the comments of E1, E4, AE1, R2 (7), R3 (1), MORE.

\item We have further included Cedrid Traveletti as co-author. He has contributed on deriving statistical properties of the methods and on improving the code for open-source quality. 

\end{enumerate}
\par \vspace{1em}

Note that the reference to page $X$ is made relative to
the original submitted manuscript given to the reviewers.


%======================================================================
\section*{Editor}
%======================================================================
All changes requested from the editor are listed below and answered on a point-by-point basis.

%\begin{answers}
	
%	\item{Comment:}\label{ec1}
\vspace{5mm}
\noindent {\bf{Comment:}}

Thank you for submitting your manuscript to The Annals of 
Applied Statistics.  Three referees, an Associate Editor (AE),
and I have read it carefully; reports from all four reviewers,
including particularly constructive reports from Referee 1
(R1R1.pdf; marked manuscript in R1R2.pdf) and the AE (R1AE1.pdf), 
can be downloaded via the links below. We all agree that your 
manuscript offers a potentially useful approach to estimating 
plume boundaries and a sequential design for the data collection
by autonomous underwater vehicles (AUVs). Citing from Referee 3
(R1E1.txt), "the authors have motivated the application well."
In this sense, the paper is consistent with the mission of AOAS.

However, the reports also identify serious shortcomings that 
prevent me from accepting your manuscript for publication at
this time. The reports are rather consistent in the nature of 
the problems that would need to be addressed for an AOAS publication.  
Briefly, these issues include:

%\item{1. Intended audience:}\label{r3c1}
\vspace{5mm}
\noindent {\bf{Main comment 1. Intended audience:}}

As you know from reading the Manuscript Submission page of our
web site (from where you submitted your manuscript), "AOAS is 
intended to be read by all professional statisticians, as well 
as by other scientists who are interested in particular areas 
of statistical application. It is important for the abstract
and introduction to say: Why the topic is of interest, and What 
is to be shown, in as non-technical a manner as possible."
Remarks from the AE (point 1), Referee 1 (point 4), and Referee 2 
(R1R3.pdf, 1) refer to this question: "Who is the intended
audience?"  Quoting from Referee 1, "I am not clear about the 
intended audience for this paper.  Or to put it another way, 
for whom is this paper important?" The reports have some 
suggestions that may guide you in making the necessary critical 
revisions along these lines.  

\reply: The comment by you, AE and reviewers made us realize that the initial submission had a motivation which was too applied. We were not selling the paper contributions in early enough, and it was not targeted sufficiently to professional statisticians. In the revised version we motivate more for statisticians. (See also comment 4 in the list of main changes above.)

\action: Our actions are to i) rewrite Introduction and Background sections to catch our intended audience and highlight our contributions, ii) restructure the presentation to highlight both statistical contributions as well as the impact such technologies and applied questions could provide for statisticians. 

%\item{2. Novelty:}\label{r3c2}
\vspace{5mm}
\noindent {\bf{Main comment 2. Novelty:}}

As noted in the AOAS mission statement, papers are expected
to include some degree of "methodological innovation."
In separate comments to the Editors, one reviewer noted,
"Methodological novelty is an issue with this paper.  Much of 
it is either standard Gaussian process calculations/formulae 
or sequential designs that were, in all but one case, previously 
proposed. Two things that could add a lot would be good code 
that others could run for the simulation studies like the authors' 
(if someone else were contemplating a similar experiment), and 
good code for finding the optimum design. That could lead to a 
fairly substantial impact of their work."

\reply: We realize that we did not stress the statistical novelty sufficiently in the submitted version. The main statistical novelty is in the closed form calculation for the design criteria of the integrated Bernoulli variance (IBV) associated with excursion sets (ESs) for multivariate processes. In the revised version this is extended to volumes (going beyond single location). We believe that part of the problem with the initial submission was the organization of these new results in the structure of the paper. This is hence re-structured in the revision.

\action: Our actions are to i) extend the closed form calculations to a criterion working on ES volume calculations (going beyond single location), ii) restructure the presentation to work directly on $p$-variate random fields and hence generalize the presentation, iii) restructure the closed form results to be presented in the main body, but with proofs in the new Appendix.

%\item{3. Assumptions }\label{r3c3}
\vspace{5mm}
\noindent {\bf{Main comment 3. Assumptions }}

Referees 1 and 2, as well as the AE, call attention to some of
the assumptions (e.g., stationary and isotropic random fields),
and ask you to address how realistic they are.

\reply: At this spatial scale there is often not so much data to infer a detailed statistical model, and this is the reason for our parsimonious construction for the real-data case. In other related situations, for instance that of larger fjord, ocean fronts, which are at a larger scale, there would usually be more data available for instance in the form of satellite data or from ensembles of computer model simulations. For such cases, we agree that it would make sense to embed more nuanced models, for instance that of larger variances at certain locations and stronger correlations in certain directions, but for our field trial here we attempted to have as few parameters as possible and had no reason to assume more complicated forms. 

That said, in the updated simulation study, which we have generalized and made open-source there are opportunities and examples of more complex covariance models (non-stationary and non-separable and non-isotropic), but we do not conduct any inference for parameters in these models from ocean model data. 

\action: Our actions include i) a better description of our parameter choices for the synthetic case and the specification for the real-data case, and ii) brief notes of the extended opportunities available for the associated open-access code, iii) references to related papers (including one of our own) that explored non-stationary Gaussian models for oceanographic phenomena at a larger scale than what is used here. 

%\item{4. Exposition}\label{r3c4}

\vspace{5mm}
\noindent {\bf{Main comment 4. Exposition}}

Both the AE and Referee 1 strongly advise shortening the first
five pages - or, at least, getting to the point (currently on p6)
much sooner.  Quoting from  Referee 1, 5: "I slogged through the 
first six [pages] before I discovered what the paper was to be 
about, on page 6."  You do not want to lose your readers, so I
advise you to engage them early on.  All four reviewers note
potential technical or typographical errors, so please proofread
carefully.

\reply: Based on the comments by yourself, AE and referees, we realize that the motivation part of the paper was too long. 

\action: Our actions include a much shorter Section 2 with background. We now aim to engage readers earlier on with the applied statistical focus and methodological contribution of our work. 

%\end{answers}


%======================================================================
 \section*{Associate Editor}
%======================================================================
All changes requested from the editor are listed below and answered on a point-by-point basis.

\setcounter{reviewer}{1}

%\begin{answers}

%\item{Comments to the Author}\label{ec}

%\aecom: Specific comments:

%\begin{itemize}[noitemsep,topsep=0pt,parsep=0pt,partopsep=0pt]

%\item[1.1.1]\label{ec1} 
\vspace{5mm}
\noindent \aecom 

General: The inferential problem considered is the paper is that of estimating the “boundary” of a plume. However, the authors never define exactly what a “boundary” is. Such a definition should be provided, as well as some discussion of why determining the location of a boundary is so important. Why is it not sufficient to create a map of predictions (with an accompanying map of prediction error variances) for the variable(s) under study throughout the region of interest (a problem for which there is a huge literature on sampling design)?\par

\reply 

\action: Added references \cite{adler2000excursion,adler2009random}.

\vspace{1em}

%\item[1.1.2]\label{ec2} 
\vspace{5mm}
\noindent \aecom 

Page 8, second paragraph of Section 4.1: The assumptions of separability and isotropy, and even stationarity, are not really necessary for the methodological development of Sections 4–6. I suggest that you wait to make these assumptions until you get to the simulation studies and example.\par

\reply We agree that they are not necessary. In the open-source that is now available, we implement non-stationary, non-separable and non-isotropic covariance models. But we do not conduct any inference for parameters. This might be possible for abundant data, but not in our small-scale resolution example. 

\action: No description of modeling assumptions in the methodological sections. A better description of our parameter choices for the synthetic case and the specification for the real-data case. Brief notes of the extended opportunities available for the associated open-access code. References to related papers (including one of our own) that explored non-stationary Gaussian models for oceanographic phenomena at a larger scale than what is used here. 

\vspace{1em}

%\item[1.1.3]\label{ec3} 
\vspace{5mm}
\noindent \aecom 

Page 9, second paragraph of Section 4.2: It has been assumed here that the measurement errors for temperature and salinity at any given location are uncorrelated. Is this a reasonable assumption? This could easily be checked by taking replicate measurements at each site.\par

\reply 

\action: The assumptions are conditionally independent measurements of salinity and temperature. We believe this is realistic as they are gathered from different sensors. In the latent Gaussian random field, the variables are dependent. 
\vspace{1em}

%\item[1.1.4]\label{ec4} 
\vspace{5mm}
\noindent \aecom 

General: Related to the previous point, I do not understand why sampling locations are being restricted to a regular grid. In fact, it seems unwise because it is well known (see, for example, Chapters 9 and 22 of Handbook of Environmental and Ecological Statistics, 2019, CRC Press) that the quality of inferences for spatial fields can be improved substantially by taking some
observations very close to others.\par

\reply

\action: The AUV and the sampling happens regularly at 1 Hz, the collected measurements are then assimilated to this regular grid. The sampling density is therefore higher than the grid resolution. Still, there is a limitation on restricting the sampling locations to finite number of fixed locations. However, in order to handle the computational load and resources available on the AUV this simplification is necessary. % Comment Trygve: Should we add some more here? 
\vspace{1em}

%\item[1.1.5]\label{ec5} 
\vspace{5mm}
\noindent \aecom 

Page 13, last line: Specifically, what was the design? We aren’t even told what the sample size ($n_y$) was used in these calculations.\par

\reply

\action: Our actions
\vspace{1em}

%\item[1.1.6]\label{ec1} 
\vspace{5mm}
\noindent \aecom 

Page 18, first paragraph of Section 7.1: The numerical values of the regression parameters given here are not meaningful without knowing the units of “West.”\par

\reply

\action: Our actions. Clarify. In a larger scale this could be improved by having auxiliary data for instance from ocean models or satellite data.

\vspace{1em}

%\item[1.1.7]\label{ec7} 
\vspace{5mm}
\noindent \aecom 

Page 24, first paragraph of Fig. 8: Please define “the modeled quadratic form of the residuals” more precisely. Perhaps a mathematical expression is needed for clarity.\par

\reply 

\action: Our actions
\vspace{1em}

%\end{itemize}


%\item{Minor comments to the Author}\label{emc}

\vspace{5mm}
\noindent {\bf{Minor AE comments to the Authors}}
%\reply: 
\begin{itemize}[noitemsep,topsep=0pt,parsep=0pt,partopsep=0pt]

%\item[1.2.1]\label{emc1} 
\item Page 3, line 19: Change “bi-variate” to “bivariate”. The same change is needed on the next-
to-last line of Page 12.

\action: We have changed this in the text - page 3 and 12.

%\item[1.2.2]\label{emc2} 
\item Page 9, three lines after (10): I presume that all off-diagonal elements of R are zeros, but this
should be stated explicitly.

\action: We have changed the statement on page 9 to: ``$\bR$ contains zeroes except on the diagonal entries, which are set to $r^2_T$ and $r^2_S$, holding the measurement error variances of temperature and salinity observations."

%\item[1.2.3]\label{emc3} 
\item Page 19, fourth line from bottom: Change ``as also" to ``as well as"

\action: We have changed this in the text - page 19.

%\item[1.2.4]\label{emc4} 
\item Page 30, second and third references listed: ``et al." is listed after the authors’ names for each
of these two references, but actually there are no additional authors beyond the two listed.

\action: we have made this change to the citations \cite{genton2015cross} and \cite{french2016credible} - page 30.
\end{itemize}


%=====================================================================
\section*{Answers to referee 1}
%=====================================================================
All changes requested from Reviewer 1 are listed below and answered on a point-by-point basis.

\setcounter{reviewer}{1}

\begin{answers}

%----------------------------------------------------------------------
\item{\revcom:  4}\label{r1c4}

Adjustments and improvements towards the intended audience?

\action: Our action

%\begin{quote}
%\frevcomment{a change in the text}
%\end{quote}

%\begin{quote}
%  \frevcomment{another change in the text} 
%\end{quote}

\item{\revcom: 4 cont.}\label{r1c4}

On including relevant code.

\action: We have made an open-source repository available for the public. This includes commented code examples for exploration and analysis of the material presented in the paper. The code is optimized for speed to allow fast computation using Pytorch (citation). The code is available here: (link). 

\item{\revcom: 5}\label{r1c5}

On shortening the paper and moving important sentences earlier. Reduced oceanographic background, while highlighting elements that reviewers did not understand about the AUV planning, waypoint graph, etc.

\action: Our actions

\item{\revcom: 6}\label{r1c6}

It is not clear why the binary classification and excursion sets are being used. Is there really a sharp boundary at the edge of a plume? Why not mapping the salinity and temperature fields instead? After all, on page 19, “achieving good predictive performance” is mentioned as a desirable feature of a good design. A lot of work has been done on that topic. And what about indicator kriging (along with logit regression) that generalizes indicator kriging that has been used to identify plumes? Modelling the p(x) is more like prohibit analysis-what is the advantage.

\action: Add reference to Adler. Motivate environmental sensing. Plants need BOTH oxygen and sun. Medical doctors diagnose with MORE THAN ONE symptom, not enough to have a fever. Plankton and sea life (relate to lice in salmon industry) need high temperatures and high salinity. 

\item{\revcom: 7}\label{r1c7}

 The sequential sampling approaches are not novel. they are mentioned as examples in the cited paper by Bect et al. What is novel is the study of how all these methods work first in simulation studies as well as well as in the case study. That work will be a valuable contribution.

\action: Our actions

\item{\revcom:  8}\label{r1c8}

While I can see some advantages of simplicity in the IBV of Bect et al for binary data, one could have used entropy as it has information-theoretical credentials. Furthermore it could cover design both for spatial mapping as well as indicator mapping. Finally model uncertainty could be included in the overall assessment of uncertainty. The more general issue here is what objective function should be optimized in the design? Are we in a multi-attribute criterion situation?

\action: Our actions. The traditional entropy calculations for this GP aim to reduce the log determinant of the updated covariance matrix at the grid locations. It will not vary with the data realizations, so in this setting this information criteria will always give the same path. The IBV criterion is aligned with our ES motivation for this study. When focusing on the expected IBV, the paths will vary with the data as we see in our simulation study and in the real-data example. We see this as a clear benefit of the approach compared with entropy-based calculations. Still, sampling approaches using hybrid criteria or multi-attribute criteria (Multi-objective optimization and Pareto fronts) might balance goals in such situations, but we leave that for future research (closing remarks). 

\item{\revcom:  9}\label{r1c9}

On the issue of covariance separability, objections have always been raised when I have used it as its unrealistic. My rejoinder has always worked, that of citing the “all models are wrong”, George Box’s famous phrase. But what I was able to show in each case is a cross-validatory assessment that showed the model was “useful”. I wonder if such a model assessment could be made here to support the assumption of separability and other things as well. By the way, in spatial mapping at least, point predictors are pretty robust against mis-specification of the covariance. Where things go sideways is in the confidence bands or ellipsoids they produce. What are the error bands when modelling binary responses? Surely the predicted plumes have fuzzy boundaries. 

\action: Our actions. We focus on point-wise predictions in the criterion. Our trends are simple, so another covariance might be overfitting too much to data. We could try to fit different ranges to temperature and salinity, while maintaining separability. Could also refer to non-separable models, but pulls focus away from the current paper's focus. There are even tests for separability, but not worth so much, what do you do if you reject? Spatio-temporal variations are more interesting here, e.g. \cite{sigrist2015stochastic,richardson2017sparsity}. Ocean models, though rarely available on the scale of river plumes, might be highly useful for training - more complex trends and covariance model could then maybe be fit from the data. 

\item{\revcom:  10}\label{r1c10}

The strategy at the bottom of page 17 seems ad hoc. Could you not use the data collected with the two ahead design to get better estimates of parameters and then with these parameters and thereafter a better myopic design?

\action: LA improved in revision! + code. %Also studied benefits of look-ahead strategies in sequential sampling in \cite{martinelli2013dynamic}.

\item{\revcom:  11}\label{r1c11}

Are these results generalizable? In other words, how much plume-to-plume variability is there? One of the problems with modelling using a lot of parameters, is there instability over replicate experiments. To put it another way which model parameters represent plume-population parameters and which represent random plume effects?

\action: This is a simple model. Ocean models for training? 

\item{\revcom:  12}\label{r1c12}

The separability assumption is to be relaxed in future work. Should publication of this case study await the proposed improvement? It would make for a much better paper.

\action: As pointed out above, we believe that spatio-temporal variations would be a more realistic extension, especially if repeated satellite data or ocean modeling is available. But this is likely relevant at a larger scale, such as that of fronts in a fjord or ocean interface. The methodological approach for sampling designs would however still be the same and this is the main contribution of the current paper. 

\item{Reviewer specific comments}\label{q11}

\reply: See the numbered list below:
\begin{itemize}[noitemsep,topsep=0pt,parsep=0pt,partopsep=0pt]

\item[1.1.1] “joint excursion sets” in the title would be a turnoff for most statisticians. Having read the paper pretty carefully I still am not sure why they are so labelled.\par
\action: Our actions
\vspace{1em}

\item[1.1.2] Why on page 13 were those particular values chosen for the means $\mu$? Would a different choice have made a difference?\par
\action: Preliminary runs set these trends. In a larger scale, one could here use ocean model inputs or satellite data. 

\vspace{1em}

\item[1.1.3] I don’t think Figure 3 conveys too much additional benefit and eliminating it for space might be considered.\par
\action: Our actions
\vspace{1em}

\item[1.1.4] Why are the ES’s called ”excursion sets? Why excursion probabilities?\par
\action: Our actions
\vspace{1em}

\item[1.1.5] I did not spot a definition for the “waypoint graph” a phrase used several times in the paper. Again points to the need to identify the audience. I doubt that most statisticians would know that phrase.\par
\action: Clarified. 
\vspace{1em}

\item[1.1.6] In statistics journals, random variables are commonly labelled by Roman characters e.g. Z, parameters in Greek e.g. $\xi$. I realize that in other scientific domains that distinction is not always made but I do recommend the change if the paper is to appear in a stats journal.\par 
\action: Our actions
\vspace{1em}

\item[1.1.7] There is not much of novelty in the Gaussian Process theory used in the paper- its pretty much standard. I suggest that the derivations stated in detail be summarized by just the results. Statisticians will know these things and users will not care.\par
\action: Our actions
\vspace{1em}

\end{itemize}

\end{answers}

% \newpage

%======================================================================
\section*{Answers to referee 2}
%======================================================================
All changes requested from Reviewer 2 are listed below and answered on a point-by-point basis.

\setcounter{reviewer}{2}

\begin{answers}
\item{\revcom:  1 - Details on AUVs}\label{r2c1}

I know very little about the AUVs you discuss, and I suspect many of your readers will be in a similar position. Additional contextual information would be helpful

\begin{itemize}[noitemsep,topsep=0pt,parsep=0pt,partopsep=0pt]

\item[2.1.1] What sort of onboard computing do they have? It may not be something like an Intel 2.1 GHz, 4-thread processor, but a more informative description of something most readers relate to would be helpful. Similar information related to RAM and hard drive space would be helpful. What are the computational limits of these machines?\par
\action: Our actions
\vspace{1em}

\item[2.1.2] How long do the AUVs typically explore in one go? Is there a minimum/maximum time?\par 
\action: Max speed is limit for coverage. Time constraints for operations related to costs of man-hours in operation. Battery capacity is not a challenge for this AUV. 
\vspace{1em}

\item[2.1.3] What kind of movements can the AUVs make? Chess-like moves (horizontal, vertical, diagonal)?\par  
\action: We clarified the descriptions of a way-point graph.
\vspace{1em}

\item[2.1.4] Is there a minimum/maximum number of points you want to sample?\par
\action: Assuming a fixed time of sampling.
\vspace{1em}

\item[2.1.5] What is the size of the grid that is typically considered? i.e., what is a typical value of $n$ in Eq.(9)?\par 
\action: Only surface considered here. No 3D with depth. 
\vspace{1em}

\item[2.1.6] Does the AUV use a selection algorithm from the very beginning or is there a minimum number of initial measurements that are made before the optimization algorithm begins?\par
\action: Batch initialization. Then adaptive sampling.
\vspace{1em}

\end{itemize}

\item{\revcom:  2}\label{r2c2}

p. 9, Eq. (8): the dependency on $\bx$ and $\bx'$ has been omitted for some of the parameters without mention. Or are the parameters constant across locations?

\action: Our actions

\item{\revcom:  3}\label{r2c3}

p. 9, How reasonable is the assumption of stationary and isotropic random fields? You may be limited by the computational abilities of the AUVs so that more complex models are computationally infeasible. That is a valid reason to choose a simpler, less accurate model. I’m just wondering if that is the reason here. 

\action: No data to fit a complex model. See Ocean model comments to Rev 1.

\item{\revcom:  4}\label{r2c4}

p. 9, Eq. (10). Are the temperature and salinity processes at a specific location effectively independent of one another? (I truly don’t know). That is what is implied by Eq. (10).

\action: Conditionally independent measurements.

\item{\revcom:  5}\label{r2c5}

p. 9, Eq. (11). The form of the equation is reminiscent of the Sherman-Morrison-Woodbury solution for fixed-rank kriging. $\bSigma$ may not be small here, but I wonder if there may be a computational advantage to considering an alternative computational solution.  See e.g., Cressie, N. and Johannesson, G. (2008), Fixed rank kriging for very large spatial data sets. Journal of the Royal Statistical Society: Series B (Statistical Methodology), 70: 209-226. doi:10.1111/j.1467-9868.2007.00633.x.

\action: Only the surface is considered here, so computational requirements are not mainly driven by the size of the domain, but rather by the number of combinations for sampling designs (computational complexity). No 3D with depth, which would require simplified spatial models like fixed-rank-kriging or similar approaches. 

\item{\revcom:  6}\label{r2c6}

p. 9, Eq. (13). Why does $\bm$ have a distribution before seeing the data? I can see how $\bm(\by)$ would have a distribution since it depends on $\by$. It almost feels Bayesian, but nothing indicates that this analysis is being pursued from a Bayesian perspective. Why is it valid for $\bm$ to have a distribution prior to seeing data?

\action: Appendix? 

\item{\revcom:  7}\label{r2c7}

p. 11, Eq. (17). I’m not understanding the advantage to conditioning on $\bm$ instead of $\by$? $\bm$ is a function of $\by$. The probability is the same two-dimensional integral either way. It seems to be related to the standardization that results in $\bZ$? Please clarify.

\action: Appendix? 

\item{\revcom:  8}\label{r2c8}

p. 11, Is $\bpi(\bm)$ the density associated with the distribution in Eq. (14)? 

\action: Appendix? 

\item{\revcom:  9}\label{r2c9}

p. 11, Is it purely for stylistic reasons that $P(\xi_T(\bx) \leq t_T, \xi_S(\bx) \leq t_S |\bm)P(\xi_T(\bx) \leq t_T, \xi_S(\bx) \leq t_S |\bm)$ isn’t written as $P(\xi_T(\bx) \leq t_T, \xi_S(\bx) \leq t_S |\bm)^2$? It seems to be a more natural way of expanding Eq. (18). 

\action: Appendix? 

\item{\revcom:  10}\label{r2c10}

p. 11, It is stated that $\bZ$ is chosen to be independent of $\bm$, so presumably the joint CDF $F_{z,m}=F_{Z}F_m$ which would certainly be convenient, but is not obvious to me why this would be true. 
Instead, it appears that $\bZ$ is chosen to be a pivotal quantity to get the simple CDF in Eq. (20). Please clarify.

\action: Appendix? 

\item{\revcom:  11}\label{r2c11}

p. 11, Eq. (21). The trailing $p(\bm)$ should be $\bpi(\bm)$. Should the right side of the equality be squared? 

\action: Our actions

\item{\revcom:  12}\label{r2c12}

This method relies on computing the marginal probabilities in each grid cell. This is computationally convenient and is likely required because of the limited computing power of the AUVs. However, the optimization should technically be performed with respect to the joint distribution across all grid cells. While the context may not allow this in practice, it would be helpful to get an idea of how much error is introduced by this approximation. I recommend creating a small but realistic-as-feasible simulation study comparing the accuracy of the probabilities, a comparison of the computational expense, and examining if/how navigation decisions change based on whether one uses the marginal/joint computations. 

\action: Joints would be too demanding here. There is more of a question of the longer planning horizon for the AUV to improve the sampling design, so the domain can be more efficiently explored. 

\item{\revcom:  13}\label{r2c13}

Fig. 7 is difficult to understand. The goal is to show the various paths taken by the AUV in the
simulation example. But it is very difficult to determine which paths are very common and which
are uncommon. Some ideas that may make this graphic more useful:

\begin{quote}
-Don’t worry about the paths but focus on the nodes visited. Represent each node using a circle or hexagon.\par
-Make the node shading darker depending on whether that node is visited in a
simulation. Nodes with no visits would be white. Nodes with 100 visits would be black.\par
-The true excursion set is a random variable, but it would be informative to overlay the
average position of the excursion set over each of the panels so the reader can see how
the simulated AUV moves back and forth across the excursion set.
\end{quote}

\action: Our actions

\item{\revcom:  14}\label{r2c14}

While the context doesn’t allow full reproducibility of the paper’s results, no implementation details are provided. What software was used to perform the bulk of analysis? Are there scripts available that can be adapted by others for similar purposes? Please provide resources for the practical implementation of the methodology discussed in the paper.

\action: Code is improved in the revision and provided for open-source usage.

\item{\revcom: Minor comments}\label{r2cminor}

\begin{itemize}[noitemsep,topsep=0pt,parsep=0pt,partopsep=0pt]

\item[2.15.1] “closed-form” is sometimes written as “closed form” (e.g., p. 7, p. 18)
\action:

\item[2.15.2] p. 9, `1s` should be ‘1s’. Similar for 0s.
\action:

\item[2.15.3] p. 15, my manuscript has -on where to sample-, which is an unusual use of dashes.
\action:
\end{itemize}

\end{answers}

%======================================================================
\section*{Answers to referee 3}
%======================================================================
All changes requested from Reviewer 3 are listed below and answered on a point-by-point basis.

\setcounter{reviewer}{3}

\begin{answers}

\item{\revcom : Major 1:}\label{r3c1}

I think this is an interesting paper. The paper is well written, 
and the authors have motivated the application well. 
The authors indicate several paths forward for work in this area, which I think pose interesting questions. They may want to elaborate a bit more on those.

\action: Space-time \cite{sigrist2015stochastic,richardson2017sparsity}. More complex sampling Look-ahead, going further. 3D sampling, requiring more computationally efficient spatial statistical models. 

\item{\revcom : Minor 1}\label{r3c2}

\reply: I think the equation $V_{m,upd}$ needs a j as a part of that subscript.

\action: Our action

\item{\revcom : Major 2}\label{r3c3}

\reply: Figure 4: Why not show excursion probabilities for, say, myopic selection as well?

\action: Our actions

\end{answers}


% == Adding references
\footnotesize
\bibliographystyle{apalike}
\bibliography{ref}

%=====================================================================
\end{document}
%=====================================================================
