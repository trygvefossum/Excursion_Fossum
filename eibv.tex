This section is devoted to computing the expected effect of the inclusion of new observations on the $\emv$ and $\ibv$ of the excursion set $\es$. Let us consider the same setting as in Section \ref{sec:cokriging} where $n$ batches of measurements have already been performed, and let 
$\currentExp{.}$ and $\currentProba{.}$ denote conditional expectation
and probability conditional on the first $n$ batches of observations, respectively. We want to co pute the effect of the inclusion of a new set of observations at $\bm{x}_{n+1}$ on the $\emv$ and $\ibv$.

For uncertainty functional such as $\ibv$ and $\emv$, we use a similar indexing system
with respect to a sequence of batch, and further indicate when applicable the new batch
of generalized location between parentheses; e.g., $\emv_{[n+1]}(\bm{x})$ denotes the value
of $\emv$ (a random variable, seen from ``step'' $n$, i.e. when knowing $Z$ solely at
the $n$ first generalized location batches) when adding observations at $\bm{x}$ to those
already available at step $n$.

\begin{propo}
\label{propo_eibv}
Let $\eibv_{[n]}(\bm{x})=\currentExp{\ibv_{[n+1]}(\bm{x})}$ (where $\bm{x}$ plays
the role of $\bm{x}_{n+1}$) be the expected effect of adding an observation a $\bm{x}$ on the IBV.
Then we have
\begin{equation}
\begin{split}
\eibv_{[n]}(\bm{x})
&=\int_{D} \varPhi_{\no}\left(\bt;~\currentMean{\uu}, \currentCov{u, u}\right) d\mes(u)\\
&-\int_{D} \varPhi_{2\no}
\left(
\left(
\begin{matrix}
\bt-\currentMean{u}\\
\bt-\currentMean{u}
\end{matrix}
\right);
\mathbf{\Sigma}_{[n]}(\uu)
\right)
d\mes(u),
\end{split}
\end{equation}
where the matrix $\mathbf{\Sigma}_{[n]}(\uu)$ is defined as
\begin{equation*}
\begin{split}
\mathbf{\Sigma}_{[n]}(\uu)&=
\left(
\begin{matrix}
\currentCov{u, u} & \currentCov{u, u}-\futureCov{u, u}\\
\currentCov{u, u}-\futureCov{u, u} & \currentCov{u, u}
\end{matrix}
%\begin{matrix}
%\currentCov{u, u} & \Delta_{[n]}(\uu)\\
%\Delta_{[n]}(\uu) & \currentCov{u, u}
%\end{matrix}
\right).\\
\end{split}
\end{equation*}
\end{propo}

As for the expected excursion measure variance, a similar result may be derived.
\begin{propo}
\label{propo_emv}

\begin{equation*}
\begin{split}
\eemv_{[n]}(\bm{x})
&=\int_{D^2} 
\varPhi_{2\no}
\left(
(\bt, \bt); \mu((u,v)), 
K((u,v),(u,v))
\right) 
\
\mathrm{d}\mes^{\otimes} %\mes 
%\productMeasure 
(u,v)\\
&-\currentExp{
    \varPhi_{\no}\left(a_1+B_1 V; C_1\right)
    \varPhi_{\no}\left(a_2+B_2 V; C_2\right)
},
\end{split}
\end{equation*}

with $V=\gp[\bm{x}_{n+1}]-\currentMean{\bm{x}_{n+1}} \sim \mathcal{N}(0_{q_{n+1}},k_{[n]}(\bm{x}_{n+1},\bm{x}_{n+1}))$ and $a_1=\bt-\currentMean{\uu}$,
$B_1=-\lambda_{[n+1,n+1]}(\uu)^T$, $a_2=\bt-\currentMean{\vv}$, $B_2=-\lambda_{[n+1,n+1]}(\vv)^T$.

This integrand finally boils down to an evaluation of the $2\no$-dimensional Gaussian CDF by
Proposition~\ref{propo3}. Details are omitted here for brevity as we did not implement further
this criterion that requires numerical integration over $\domain^2$.
\end{propo}


Note that the calculation of the multivariate Gaussian cumulative distribution
may be done effectively using code such as that of
\cite{genz2009computation}. 

\textcolor{red}{Do we really want to keep this?}
A critical element in the derivation of a closed-form for the EIBV is
that the conditional mean in the cokriging equations is a linear
(affine) function of the data $\by_d$ and the covariance is not a
function of the data. This in turn means that the probabilities in Eq.
\eqref{eq:post_ep} involve inequality statements for linear
combinations of Gaussian variables. Related closed-form solutions have
been noted in similar contexts \citep{bhattacharjya2013value,
  chevalier2014fast,stroh}, but not generalized to our situation with
random sets for vector-valued GPs.
